%
\begin{isabellebody}%
\setisabellecontext{StutteringDefs}%
%
\isadelimdocument
%
\endisadelimdocument
%
\isatagdocument
%
\isamarkupchapter{Properties of TESL%
}
\isamarkuptrue%
%
\isamarkupsection{Stuttering Invariance%
}
\isamarkuptrue%
%
\endisatagdocument
{\isafolddocument}%
%
\isadelimdocument
%
\endisadelimdocument
%
\isadelimtheory
%
\endisadelimtheory
%
\isatagtheory
\isacommand{theory}\isamarkupfalse%
\ StutteringDefs\isanewline
\isanewline
\isakeyword{imports}\ Denotational\isanewline
\isanewline
\isakeyword{begin}%
\endisatagtheory
{\isafoldtheory}%
%
\isadelimtheory
%
\endisadelimtheory
%
\isadelimdocument
%
\endisadelimdocument
%
\isatagdocument
%
\isamarkupsubsection{Definition of stuttering%
}
\isamarkuptrue%
%
\endisatagdocument
{\isafolddocument}%
%
\isadelimdocument
%
\endisadelimdocument
%
\begin{isamarkuptext}%
A dilating function inserts empty instants in a run.
  It is strictly increasing, the image of a \isa{nat} is greater than it,
  no instant is inserted before the first one 
  and if n is not in the image of the function, no clock ticks at instant n.%
\end{isamarkuptext}\isamarkuptrue%
\isacommand{definition}\isamarkupfalse%
\ dilating{\isacharunderscore}fun\isanewline
\isakeyword{where}\isanewline
\ \ {\isacartoucheopen}dilating{\isacharunderscore}fun\ {\isacharparenleft}f{\isacharcolon}{\isacharcolon}nat\ {\isasymRightarrow}\ nat{\isacharparenright}\ {\isacharparenleft}r{\isacharcolon}{\isacharcolon}{\isacharprime}a{\isacharcolon}{\isacharcolon}linordered{\isacharunderscore}field\ run{\isacharparenright}\isanewline
\ \ \ \ {\isasymequiv}\ strict{\isacharunderscore}mono\ f\ {\isasymand}\ {\isacharparenleft}f\ {\isadigit{0}}\ {\isacharequal}\ {\isadigit{0}}{\isacharparenright}\ {\isasymand}\ {\isacharparenleft}{\isasymforall}n{\isachardot}\ f\ n\ {\isasymge}\ n\isanewline
\ \ \ \ {\isasymand}\ {\isacharparenleft}{\isacharparenleft}{\isasymnexists}n\isactrlsub {\isadigit{0}}{\isachardot}\ f\ n\isactrlsub {\isadigit{0}}\ {\isacharequal}\ n{\isacharparenright}\ {\isasymlongrightarrow}\ {\isacharparenleft}{\isasymforall}c{\isachardot}\ {\isasymnot}{\isacharparenleft}hamlet\ {\isacharparenleft}{\isacharparenleft}Rep{\isacharunderscore}run\ r{\isacharparenright}\ n\ c{\isacharparenright}{\isacharparenright}{\isacharparenright}{\isacharparenright}\isanewline
\ \ \ \ {\isasymand}\ {\isacharparenleft}{\isacharparenleft}{\isasymnexists}n\isactrlsub {\isadigit{0}}{\isachardot}\ f\ n\isactrlsub {\isadigit{0}}\ {\isacharequal}\ {\isacharparenleft}Suc\ n{\isacharparenright}{\isacharparenright}\ {\isasymlongrightarrow}\ {\isacharparenleft}{\isasymforall}c{\isachardot}\ time\ {\isacharparenleft}{\isacharparenleft}Rep{\isacharunderscore}run\ r{\isacharparenright}\ {\isacharparenleft}Suc\ n{\isacharparenright}\ c{\isacharparenright}\ {\isacharequal}\ time\ {\isacharparenleft}{\isacharparenleft}Rep{\isacharunderscore}run\ r{\isacharparenright}\ n\ c{\isacharparenright}{\isacharparenright}{\isacharparenright}\isanewline
\ \ \ {\isacharparenright}{\isacartoucheclose}%
\begin{isamarkuptext}%
Dilating a run. A run \isa{r} is a dilation of a run \isa{sub} by function \isa{f} if:

%
\begin{itemize}%
\item \isa{f} is a dilating function on the hamlet of \isa{r} 

\item time is preserved in stuttering instants

\item the time in \isa{r} is the time in \isa{sub}  dilated by \isa{f}

\item the hamlet in \isa{r} is the hamlet in sub dilated by \isa{f}%
\end{itemize}%
\end{isamarkuptext}\isamarkuptrue%
\isacommand{definition}\isamarkupfalse%
\ dilating\isanewline
\ \ \isakeyword{where}\ {\isacartoucheopen}dilating\ f\ sub\ r\ {\isasymequiv}\ \ \ dilating{\isacharunderscore}fun\ f\ r\isanewline
\ \ \ \ \ \ \ \ \ \ \ \ \ \ \ \ \ \ \ \ \ \ \ \ \ \ \ \ {\isasymand}\ {\isacharparenleft}{\isasymforall}n\ c{\isachardot}\ time\ {\isacharparenleft}{\isacharparenleft}Rep{\isacharunderscore}run\ sub{\isacharparenright}\ n\ c{\isacharparenright}\ {\isacharequal}\ time\ {\isacharparenleft}{\isacharparenleft}Rep{\isacharunderscore}run\ r{\isacharparenright}\ {\isacharparenleft}f\ n{\isacharparenright}\ c{\isacharparenright}{\isacharparenright}\isanewline
\ \ \ \ \ \ \ \ \ \ \ \ \ \ \ \ \ \ \ \ \ \ \ \ \ \ \ \ {\isasymand}\ {\isacharparenleft}{\isasymforall}n\ c{\isachardot}\ hamlet\ {\isacharparenleft}{\isacharparenleft}Rep{\isacharunderscore}run\ sub{\isacharparenright}\ n\ c{\isacharparenright}\ {\isacharequal}\ hamlet\ {\isacharparenleft}{\isacharparenleft}Rep{\isacharunderscore}run\ r{\isacharparenright}\ {\isacharparenleft}f\ n{\isacharparenright}\ c{\isacharparenright}{\isacharparenright}{\isacartoucheclose}%
\begin{isamarkuptext}%
A \emph{run}  is a \emph{subrun} of another run if there exists a dilation between them.%
\end{isamarkuptext}\isamarkuptrue%
\isacommand{definition}\isamarkupfalse%
\ is{\isacharunderscore}subrun\ {\isacharcolon}{\isacharcolon}{\isacartoucheopen}{\isacharprime}a{\isacharcolon}{\isacharcolon}linordered{\isacharunderscore}field\ run\ {\isasymRightarrow}\ {\isacharprime}a\ run\ {\isasymRightarrow}\ bool{\isacartoucheclose}\ {\isacharparenleft}\isakeyword{infixl}\ {\isachardoublequoteopen}{\isasymlless}{\isachardoublequoteclose}\ {\isadigit{6}}{\isadigit{0}}{\isacharparenright}\isanewline
\isakeyword{where}\isanewline
\ \ {\isacartoucheopen}sub\ {\isasymlless}\ r\ {\isasymequiv}\ {\isacharparenleft}{\isasymexists}f{\isachardot}\ dilating\ f\ sub\ r{\isacharparenright}{\isacartoucheclose}%
\begin{isamarkuptext}%
A \isa{tick{\isacharunderscore}count\ r\ c\ n} is a
  number of ticks of clock \isa{c} in run \isa{r} upto instant \isa{n}.%
\end{isamarkuptext}\isamarkuptrue%
\isacommand{definition}\isamarkupfalse%
\ tick{\isacharunderscore}count\ {\isacharcolon}{\isacharcolon}\ {\isacartoucheopen}{\isacharprime}a{\isacharcolon}{\isacharcolon}linordered{\isacharunderscore}field\ run\ {\isasymRightarrow}\ clock\ {\isasymRightarrow}\ nat\ {\isasymRightarrow}\ nat{\isacartoucheclose}\isanewline
\isakeyword{where}\isanewline
\ \ {\isacartoucheopen}tick{\isacharunderscore}count\ r\ c\ n\ {\isacharequal}\ card\ {\isacharbraceleft}i{\isachardot}\ i\ {\isasymle}\ n\ {\isasymand}\ hamlet\ {\isacharparenleft}{\isacharparenleft}Rep{\isacharunderscore}run\ r{\isacharparenright}\ i\ c{\isacharparenright}{\isacharbraceright}{\isacartoucheclose}%
\begin{isamarkuptext}%
A \isa{tick{\isacharunderscore}count{\isacharunderscore}strict\ r\ c\ n} is a number of ticks of clock \isa{c} in run 
      \isa{r} upto but  excluding instant \isa{n}.%
\end{isamarkuptext}\isamarkuptrue%
\isacommand{definition}\isamarkupfalse%
\ tick{\isacharunderscore}count{\isacharunderscore}strict\ {\isacharcolon}{\isacharcolon}\ {\isacartoucheopen}{\isacharprime}a{\isacharcolon}{\isacharcolon}linordered{\isacharunderscore}field\ run\ {\isasymRightarrow}\ clock\ {\isasymRightarrow}\ nat\ {\isasymRightarrow}\ nat{\isacartoucheclose}\isanewline
\isakeyword{where}\isanewline
\ \ {\isacartoucheopen}tick{\isacharunderscore}count{\isacharunderscore}strict\ r\ c\ n\ {\isacharequal}\ card\ {\isacharbraceleft}i{\isachardot}\ i\ {\isacharless}\ n\ {\isasymand}\ hamlet\ {\isacharparenleft}{\isacharparenleft}Rep{\isacharunderscore}run\ r{\isacharparenright}\ i\ c{\isacharparenright}{\isacharbraceright}{\isacartoucheclose}\isanewline
\isanewline
\isacommand{definition}\isamarkupfalse%
\ contracting{\isacharunderscore}fun\isanewline
\ \ \isakeyword{where}\ {\isacartoucheopen}contracting{\isacharunderscore}fun\ g\ {\isasymequiv}\ mono\ g\ {\isasymand}\ g\ {\isadigit{0}}\ {\isacharequal}\ {\isadigit{0}}\ {\isasymand}\ {\isacharparenleft}{\isasymforall}n{\isachardot}\ g\ n\ {\isasymle}\ n{\isacharparenright}{\isacartoucheclose}\isanewline
\isanewline
\isacommand{definition}\isamarkupfalse%
\ contracting\isanewline
\isakeyword{where}\ \isanewline
\ \ {\isacartoucheopen}contracting\ g\ r\ sub\ f\ {\isasymequiv}\ \ contracting{\isacharunderscore}fun\ g\isanewline
\ \ \ \ \ \ \ \ \ \ \ \ \ \ \ \ \ \ \ \ \ \ \ \ \ \ {\isasymand}\ {\isacharparenleft}{\isasymforall}n\ c\ k{\isachardot}\ f\ {\isacharparenleft}g\ n{\isacharparenright}\ {\isasymle}\ k\ {\isasymand}\ k\ {\isasymle}\ n\isanewline
\ \ \ \ \ \ \ \ \ \ \ \ \ \ \ \ \ \ \ \ \ \ \ \ \ \ \ \ \ \ {\isasymlongrightarrow}\ time\ {\isacharparenleft}{\isacharparenleft}Rep{\isacharunderscore}run\ r{\isacharparenright}\ k\ c{\isacharparenright}\ {\isacharequal}\ time\ {\isacharparenleft}{\isacharparenleft}Rep{\isacharunderscore}run\ sub{\isacharparenright}\ {\isacharparenleft}g\ n{\isacharparenright}\ c{\isacharparenright}{\isacharparenright}\isanewline
\ \ \ \ \ \ \ \ \ \ \ \ \ \ \ \ \ \ \ \ \ \ \ \ \ \ {\isasymand}\ {\isacharparenleft}{\isasymforall}n\ c\ k{\isachardot}\ f\ {\isacharparenleft}g\ n{\isacharparenright}\ {\isacharless}\ k\ {\isasymand}\ k\ {\isasymle}\ n\isanewline
\ \ \ \ \ \ \ \ \ \ \ \ \ \ \ \ \ \ \ \ \ \ \ \ \ \ \ \ \ \ {\isasymlongrightarrow}\ {\isasymnot}\ hamlet\ {\isacharparenleft}{\isacharparenleft}Rep{\isacharunderscore}run\ r{\isacharparenright}\ k\ c{\isacharparenright}{\isacharparenright}{\isacartoucheclose}\isanewline
\isanewline
\isacommand{definition}\isamarkupfalse%
\ {\isacartoucheopen}dil{\isacharunderscore}inverse\ f{\isacharcolon}{\isacharcolon}{\isacharparenleft}nat\ {\isasymRightarrow}\ nat{\isacharparenright}\ {\isasymequiv}\ {\isacharparenleft}{\isasymlambda}n{\isachardot}\ Max\ {\isacharbraceleft}i{\isachardot}\ f\ i\ {\isasymle}\ n{\isacharbraceright}{\isacharparenright}{\isacartoucheclose}\isanewline
%
\isadelimtheory
\isanewline
%
\endisadelimtheory
%
\isatagtheory
\isacommand{end}\isamarkupfalse%
%
\endisatagtheory
{\isafoldtheory}%
%
\isadelimtheory
%
\endisadelimtheory
%
\end{isabellebody}%
\endinput
%:%file=~/Documents/Recherche/Thesards/2014_Hai_NGUYEN_VAN/Heron_git/reiher2/reiher/src/StutteringDefs.thy%:%
%:%11=1%:%
%:%15=3%:%
%:%31=5%:%
%:%32=5%:%
%:%33=6%:%
%:%34=7%:%
%:%35=8%:%
%:%36=9%:%
%:%50=11%:%
%:%62=14%:%
%:%63=15%:%
%:%64=16%:%
%:%65=17%:%
%:%67=19%:%
%:%68=19%:%
%:%69=20%:%
%:%70=21%:%
%:%76=28%:%
%:%80=29%:%
%:%82=30%:%
%:%84=31%:%
%:%86=32%:%
%:%89=34%:%
%:%90=34%:%
%:%91=35%:%
%:%95=39%:%
%:%97=40%:%
%:%98=40%:%
%:%99=41%:%
%:%100=42%:%
%:%102=44%:%
%:%103=45%:%
%:%105=47%:%
%:%106=47%:%
%:%107=48%:%
%:%108=49%:%
%:%110=51%:%
%:%111=52%:%
%:%113=53%:%
%:%114=53%:%
%:%115=54%:%
%:%116=55%:%
%:%117=56%:%
%:%118=57%:%
%:%119=57%:%
%:%120=58%:%
%:%121=59%:%
%:%122=60%:%
%:%123=60%:%
%:%124=61%:%
%:%125=62%:%
%:%129=66%:%
%:%130=67%:%
%:%131=68%:%
%:%132=68%:%
%:%135=69%:%
%:%140=70%:%