%
\begin{isabellebody}%
\setisabellecontext{TESL}%
%
\begin{isamarkuptext}%
\chapter[Core TESL: Syntax and Basics]{The Core of the TESL Language: Syntax and Basics}%
\end{isamarkuptext}\isamarkuptrue%
%
\isadelimtheory
%
\endisadelimtheory
%
\isatagtheory
\isacommand{theory}\isamarkupfalse%
\ TESL\isanewline
\isakeyword{imports}\ Main\isanewline
\isanewline
\isakeyword{begin}%
\endisatagtheory
{\isafoldtheory}%
%
\isadelimtheory
%
\endisadelimtheory
%
\isadelimdocument
%
\endisadelimdocument
%
\isatagdocument
%
\isamarkupsection{Syntactic Representation%
}
\isamarkuptrue%
%
\endisatagdocument
{\isafolddocument}%
%
\isadelimdocument
%
\endisadelimdocument
%
\begin{isamarkuptext}%
We define here the syntax of TESL specifications.%
\end{isamarkuptext}\isamarkuptrue%
%
\isadelimdocument
%
\endisadelimdocument
%
\isatagdocument
%
\isamarkupsubsection{Basic elements of a specification%
}
\isamarkuptrue%
%
\endisatagdocument
{\isafolddocument}%
%
\isadelimdocument
%
\endisadelimdocument
%
\begin{isamarkuptext}%
The following items appear in specifications:

%
\begin{itemize}%
\item Clocks, which are identified by a name.

\item An instant on a clock is identified by its index, starting from 0

\item Tag constants are just constants of a type which denotes the metric time space.%
\end{itemize}%
\end{isamarkuptext}\isamarkuptrue%
\isacommand{datatype}\isamarkupfalse%
\ \ \ \ \ clock\ \ \ \ \ \ \ \ \ {\isacharequal}\ Clk\ {\isacartoucheopen}string{\isacartoucheclose}\isanewline
\isacommand{type{\isacharunderscore}synonym}\isamarkupfalse%
\ instant{\isacharunderscore}index\ {\isacharequal}\ {\isacartoucheopen}nat{\isacartoucheclose}\isanewline
\isanewline
\isacommand{datatype}\isamarkupfalse%
\ \ \ \ \ {\isacharprime}{\isasymtau}\ tag{\isacharunderscore}const\ {\isacharequal}\ \ TConst\ \ \ {\isacharparenleft}the{\isacharunderscore}tag{\isacharunderscore}const\ {\isacharcolon}\ {\isacharprime}{\isasymtau}{\isacharparenright}\ \ \ \ \ \ \ \ \ {\isacharparenleft}{\isacartoucheopen}{\isasymtau}\isactrlsub c\isactrlsub s\isactrlsub t{\isacartoucheclose}{\isacharparenright}%
\begin{isamarkuptext}%
Tag variables are used to refer to the time on a clock at a given instant index.
  Tag expressions are used to build a new tag by adding a constant delay to a tag variable.%
\end{isamarkuptext}\isamarkuptrue%
\isacommand{datatype}\isamarkupfalse%
\ tag{\isacharunderscore}var\ {\isacharequal}\isanewline
\ \ TSchematic\ {\isacartoucheopen}clock\ {\isacharasterisk}\ instant{\isacharunderscore}index{\isacartoucheclose}\ {\isacharparenleft}{\isacartoucheopen}{\isasymtau}\isactrlsub v\isactrlsub a\isactrlsub r{\isacartoucheclose}{\isacharparenright}\isanewline
\isacommand{datatype}\isamarkupfalse%
\ {\isacharprime}{\isasymtau}\ tag{\isacharunderscore}expr\ \ \ \ \ \ {\isacharequal}\ \ \isanewline
\ \ \ \ \ \ \ \ \ \ \ \ \ \ \ \ \ \ \ \ \ \ \ \ \ \ \ \ \ \ AddDelay\ {\isacartoucheopen}tag{\isacharunderscore}var{\isacartoucheclose}\ {\isacartoucheopen}{\isacharprime}{\isasymtau}\ tag{\isacharunderscore}const{\isacartoucheclose}\ {\isacharparenleft}{\isacartoucheopen}{\isasymlparr}\ {\isacharunderscore}\ {\isasymoplus}\ {\isacharunderscore}\ {\isasymrparr}{\isacartoucheclose}{\isacharparenright}%
\isadelimdocument
%
\endisadelimdocument
%
\isatagdocument
%
\isamarkupsubsection{Operators for the TESL language%
}
\isamarkuptrue%
%
\endisatagdocument
{\isafolddocument}%
%
\isadelimdocument
%
\endisadelimdocument
%
\begin{isamarkuptext}%
The type of atomic TESL constraints, which can be combined to form specifications.%
\end{isamarkuptext}\isamarkuptrue%
\isacommand{datatype}\isamarkupfalse%
\ {\isacharprime}{\isasymtau}\ TESL{\isacharunderscore}atomic\ {\isacharequal}\isanewline
\ \ \ \ SporadicOn\ \ \ \ \ \ \ \ \ {\isacartoucheopen}clock{\isacartoucheclose}\ {\isacartoucheopen}{\isacharprime}{\isasymtau}\ tag{\isacharunderscore}const{\isacartoucheclose}\ \ {\isacartoucheopen}clock{\isacartoucheclose}\ \ {\isacharparenleft}{\isacartoucheopen}{\isacharunderscore}\ sporadic\ {\isacharunderscore}\ on\ {\isacharunderscore}{\isacartoucheclose}\ {\isadigit{5}}{\isadigit{5}}{\isacharparenright}\isanewline
\ \ {\isacharbar}\ TagRelation\ \ \ \ \ \ \ \ {\isacartoucheopen}clock{\isacartoucheclose}\ {\isacartoucheopen}clock{\isacartoucheclose}\ {\isacartoucheopen}{\isacharparenleft}{\isacharprime}{\isasymtau}\ tag{\isacharunderscore}const\ {\isasymtimes}\ {\isacharprime}{\isasymtau}\ tag{\isacharunderscore}const{\isacharparenright}\ {\isasymRightarrow}\ bool{\isacartoucheclose}\ \isanewline
\ \ \ \ \ \ \ \ \ \ \ \ \ \ \ \ \ \ \ \ \ \ \ \ \ \ \ \ \ \ \ \ \ \ \ \ \ \ \ \ \ \ \ \ \ \ \ \ \ \ \ \ \ \ {\isacharparenleft}{\isacartoucheopen}time{\isacharminus}relation\ {\isasymlfloor}{\isacharunderscore}{\isacharcomma}\ {\isacharunderscore}{\isasymrfloor}\ {\isasymin}\ {\isacharunderscore}{\isacartoucheclose}\ {\isadigit{5}}{\isadigit{5}}{\isacharparenright}\isanewline
\ \ {\isacharbar}\ Implies\ \ \ \ \ \ \ \ \ \ \ \ {\isacartoucheopen}clock{\isacartoucheclose}\ {\isacartoucheopen}clock{\isacartoucheclose}\ \ \ \ \ \ \ \ \ \ \ \ \ \ \ \ \ \ {\isacharparenleft}\isakeyword{infixr}\ {\isacartoucheopen}implies{\isacartoucheclose}\ {\isadigit{5}}{\isadigit{5}}{\isacharparenright}\isanewline
\ \ {\isacharbar}\ ImpliesNot\ \ \ \ \ \ \ \ \ {\isacartoucheopen}clock{\isacartoucheclose}\ {\isacartoucheopen}clock{\isacartoucheclose}\ \ \ \ \ \ \ \ \ \ \ \ \ \ \ \ \ \ {\isacharparenleft}\isakeyword{infixr}\ {\isacartoucheopen}implies\ not{\isacartoucheclose}\ {\isadigit{5}}{\isadigit{5}}{\isacharparenright}\isanewline
\ \ {\isacharbar}\ TimeDelayedBy\ \ \ \ \ \ {\isacartoucheopen}clock{\isacartoucheclose}\ {\isacartoucheopen}{\isacharprime}{\isasymtau}\ tag{\isacharunderscore}const{\isacartoucheclose}\ {\isacartoucheopen}clock{\isacartoucheclose}\ {\isacartoucheopen}clock{\isacartoucheclose}\isanewline
\ \ \ \ \ \ \ \ \ \ \ \ \ \ \ \ \ \ \ \ \ \ \ \ \ \ \ \ \ \ \ \ \ \ \ \ \ \ \ \ \ \ \ \ \ \ \ \ \ \ \ \ \ \ {\isacharparenleft}{\isacartoucheopen}{\isacharunderscore}\ time{\isacharminus}delayed\ by\ {\isacharunderscore}\ on\ {\isacharunderscore}\ implies\ {\isacharunderscore}{\isacartoucheclose}\ {\isadigit{5}}{\isadigit{5}}{\isacharparenright}\isanewline
\ \ {\isacharbar}\ RelaxedTimeDelayed\ {\isacartoucheopen}clock{\isacartoucheclose}\ {\isacartoucheopen}{\isacharprime}{\isasymtau}\ tag{\isacharunderscore}const{\isacartoucheclose}\ {\isacartoucheopen}clock{\isacartoucheclose}\ {\isacartoucheopen}clock{\isacartoucheclose}\isanewline
\ \ \ \ \ \ \ \ \ \ \ \ \ \ \ \ \ \ \ \ \ \ \ \ \ \ \ \ \ \ \ \ \ \ \ \ \ \ \ \ \ \ \ \ \ \ \ \ \ \ \ \ \ \ {\isacharparenleft}{\isacartoucheopen}{\isacharunderscore}\ time{\isacharminus}delayed{\isasymbowtie}\ by\ {\isacharunderscore}\ on\ {\isacharunderscore}\ implies\ {\isacharunderscore}{\isacartoucheclose}\ {\isadigit{5}}{\isadigit{5}}{\isacharparenright}\isanewline
\ \ {\isacharbar}\ WeaklyPrecedes\ \ \ \ \ {\isacartoucheopen}clock{\isacartoucheclose}\ {\isacartoucheopen}clock{\isacartoucheclose}\ \ \ \ \ \ \ \ \ \ \ \ \ \ \ \ \ \ {\isacharparenleft}\isakeyword{infixr}\ {\isacartoucheopen}weakly\ precedes{\isacartoucheclose}\ {\isadigit{5}}{\isadigit{5}}{\isacharparenright}\isanewline
\ \ {\isacharbar}\ StrictlyPrecedes\ \ \ {\isacartoucheopen}clock{\isacartoucheclose}\ {\isacartoucheopen}clock{\isacartoucheclose}\ \ \ \ \ \ \ \ \ \ \ \ \ \ \ \ \ \ {\isacharparenleft}\isakeyword{infixr}\ {\isacartoucheopen}strictly\ precedes{\isacartoucheclose}\ {\isadigit{5}}{\isadigit{5}}{\isacharparenright}\isanewline
\ \ {\isacharbar}\ Kills\ \ \ \ \ \ \ \ \ \ \ \ \ \ {\isacartoucheopen}clock{\isacartoucheclose}\ {\isacartoucheopen}clock{\isacartoucheclose}\ \ \ \ \ \ \ \ \ \ \ \ \ \ \ \ \ \ {\isacharparenleft}\isakeyword{infixr}\ {\isacartoucheopen}kills{\isacartoucheclose}\ {\isadigit{5}}{\isadigit{5}}{\isacharparenright}\isanewline
%
\isamarkupcmt{The following constraints are not part of the TESL language,
    they are added only for implementing the operational semantics%
}\isanewline
\ \ {\isacharbar}\ SporadicOnTvar\ \ \ \ \ {\isacartoucheopen}clock{\isacartoucheclose}\ {\isacartoucheopen}{\isacharprime}{\isasymtau}\ tag{\isacharunderscore}expr{\isacartoucheclose}\ \ {\isacartoucheopen}clock{\isacartoucheclose}\ \ \ {\isacharparenleft}{\isacartoucheopen}{\isacharunderscore}\ sporadic{\isasymsharp}\ {\isacharunderscore}\ on\ {\isacharunderscore}{\isacartoucheclose}\ {\isadigit{5}}{\isadigit{5}}{\isacharparenright}%
\begin{isamarkuptext}%
Some constraints were introduced for the implementation of the operational semantics.
  They are not allowed in user-level TESL specification and are not public.%
\end{isamarkuptext}\isamarkuptrue%
\isacommand{fun}\isamarkupfalse%
\ is{\isacharunderscore}public{\isacharunderscore}atom\ {\isacharcolon}{\isacharcolon}\ {\isacartoucheopen}{\isacharprime}{\isasymtau}\ TESL{\isacharunderscore}atomic\ {\isasymRightarrow}\ bool{\isacartoucheclose}\ \isakeyword{where}\isanewline
\ \ \ \ {\isacartoucheopen}is{\isacharunderscore}public{\isacharunderscore}atom\ {\isacharparenleft}{\isacharunderscore}\ sporadic{\isasymsharp}\ {\isacharunderscore}\ on\ {\isacharunderscore}{\isacharparenright}\ \ \ \ \ \ \ \ \ \ \ \ \ \ \ \ \ \ {\isacharequal}\ False{\isacartoucheclose}\isanewline
\ \ {\isacharbar}\ {\isacartoucheopen}is{\isacharunderscore}public{\isacharunderscore}atom\ {\isacharunderscore}\ \ \ \ \ \ \ \ \ \ \ \ \ \ \ \ \ \ \ \ \ \ \ \ \ \ \ \ \ \ \ \ \ \ \ \ \ {\isacharequal}\ True{\isacartoucheclose}%
\begin{isamarkuptext}%
A TESL formula is just a list of atomic constraints, with implicit conjunction
  for the semantics.%
\end{isamarkuptext}\isamarkuptrue%
\isacommand{type{\isacharunderscore}synonym}\isamarkupfalse%
\ {\isacharprime}{\isasymtau}\ TESL{\isacharunderscore}formula\ {\isacharequal}\ {\isacartoucheopen}{\isacharprime}{\isasymtau}\ TESL{\isacharunderscore}atomic\ list{\isacartoucheclose}\isanewline
\isanewline
\isacommand{fun}\isamarkupfalse%
\ is{\isacharunderscore}public{\isacharunderscore}spec\ {\isacharcolon}{\isacharcolon}\ {\isacartoucheopen}{\isacharprime}{\isasymtau}\ TESL{\isacharunderscore}atomic\ list\ {\isasymRightarrow}\ bool{\isacartoucheclose}\ \isakeyword{where}\isanewline
\ \ \ \ {\isacartoucheopen}is{\isacharunderscore}public{\isacharunderscore}spec\ {\isacharbrackleft}{\isacharbrackright}\ {\isacharequal}\ True{\isacartoucheclose}\isanewline
\ \ {\isacharbar}\ {\isacartoucheopen}is{\isacharunderscore}public{\isacharunderscore}spec\ {\isacharparenleft}{\isasymphi}{\isacharhash}S{\isacharparenright}\ {\isacharequal}\ {\isacharparenleft}{\isacharparenleft}is{\isacharunderscore}public{\isacharunderscore}atom\ {\isasymphi}{\isacharparenright}\ {\isasymand}\ {\isacharparenleft}is{\isacharunderscore}public{\isacharunderscore}spec\ S{\isacharparenright}{\isacharparenright}{\isacartoucheclose}%
\begin{isamarkuptext}%
We call \emph{positive atoms} the atomic constraints that create ticks from nothing.
  Only sporadic constraints are positive in the current version of TESL.%
\end{isamarkuptext}\isamarkuptrue%
\isacommand{fun}\isamarkupfalse%
\ positive{\isacharunderscore}atom\ {\isacharcolon}{\isacharcolon}\ {\isacartoucheopen}{\isacharprime}{\isasymtau}\ TESL{\isacharunderscore}atomic\ {\isasymRightarrow}\ bool{\isacartoucheclose}\ \isakeyword{where}\isanewline
\ \ \ \ {\isacartoucheopen}positive{\isacharunderscore}atom\ {\isacharparenleft}{\isacharunderscore}\ sporadic\ {\isacharunderscore}\ on\ {\isacharunderscore}{\isacharparenright}\ {\isacharequal}\ True{\isacartoucheclose}\isanewline
\ \ {\isacharbar}\ {\isacartoucheopen}positive{\isacharunderscore}atom\ {\isacharparenleft}{\isacharunderscore}\ sporadic{\isasymsharp}\ {\isacharunderscore}\ on\ {\isacharunderscore}{\isacharparenright}\ {\isacharequal}\ True{\isacartoucheclose}\isanewline
\ \ {\isacharbar}\ {\isacartoucheopen}positive{\isacharunderscore}atom\ {\isacharunderscore}\ \ \ \ \ \ \ \ \ \ \ \ \ \ \ \ \ \ \ {\isacharequal}\ False{\isacartoucheclose}%
\begin{isamarkuptext}%
The \isa{NoSporadic} function removes sporadic constraints from a TESL formula.%
\end{isamarkuptext}\isamarkuptrue%
\isacommand{abbreviation}\isamarkupfalse%
\ NoSporadic\ {\isacharcolon}{\isacharcolon}\ {\isacartoucheopen}{\isacharprime}{\isasymtau}\ TESL{\isacharunderscore}formula\ {\isasymRightarrow}\ {\isacharprime}{\isasymtau}\ TESL{\isacharunderscore}formula{\isacartoucheclose}\isanewline
\isakeyword{where}\ \isanewline
\ \ {\isacartoucheopen}NoSporadic\ f\ {\isasymequiv}\ {\isacharparenleft}List{\isachardot}filter\ {\isacharparenleft}{\isasymlambda}f\isactrlsub a\isactrlsub t\isactrlsub o\isactrlsub m{\isachardot}\ case\ f\isactrlsub a\isactrlsub t\isactrlsub o\isactrlsub m\ of\isanewline
\ \ \ \ \ \ {\isacharunderscore}\ sporadic\ {\isacharunderscore}\ on\ {\isacharunderscore}\ {\isasymRightarrow}\ False\isanewline
\ \ \ \ {\isacharbar}\ {\isacharunderscore}\ {\isasymRightarrow}\ True{\isacharparenright}\ f{\isacharparenright}{\isacartoucheclose}%
\isadelimdocument
%
\endisadelimdocument
%
\isatagdocument
%
\isamarkupsubsection{Field Structure of the Metric Time Space%
}
\isamarkuptrue%
%
\endisatagdocument
{\isafolddocument}%
%
\isadelimdocument
%
\endisadelimdocument
%
\begin{isamarkuptext}%
In order to handle tag relations and delays, tags must belong to a field.
  We show here that this is the case when the type parameter of \isa{{\isacharprime}{\isasymtau}\ tag{\isacharunderscore}const} 
  is itself a field.%
\end{isamarkuptext}\isamarkuptrue%
\isacommand{instantiation}\isamarkupfalse%
\ tag{\isacharunderscore}const\ {\isacharcolon}{\isacharcolon}{\isacharparenleft}field{\isacharparenright}field\isanewline
\isakeyword{begin}\isanewline
\ \ \isacommand{fun}\isamarkupfalse%
\ inverse{\isacharunderscore}tag{\isacharunderscore}const\isanewline
\ \ \isakeyword{where}\ {\isacartoucheopen}inverse\ {\isacharparenleft}{\isasymtau}\isactrlsub c\isactrlsub s\isactrlsub t\ t{\isacharparenright}\ {\isacharequal}\ {\isasymtau}\isactrlsub c\isactrlsub s\isactrlsub t\ {\isacharparenleft}inverse\ t{\isacharparenright}{\isacartoucheclose}\isanewline
\isanewline
\ \ \isacommand{fun}\isamarkupfalse%
\ divide{\isacharunderscore}tag{\isacharunderscore}const\ \isanewline
\ \ \ \ \isakeyword{where}\ {\isacartoucheopen}divide\ {\isacharparenleft}{\isasymtau}\isactrlsub c\isactrlsub s\isactrlsub t\ t\isactrlsub {\isadigit{1}}{\isacharparenright}\ {\isacharparenleft}{\isasymtau}\isactrlsub c\isactrlsub s\isactrlsub t\ t\isactrlsub {\isadigit{2}}{\isacharparenright}\ {\isacharequal}\ {\isasymtau}\isactrlsub c\isactrlsub s\isactrlsub t\ {\isacharparenleft}divide\ t\isactrlsub {\isadigit{1}}\ t\isactrlsub {\isadigit{2}}{\isacharparenright}{\isacartoucheclose}\isanewline
\isanewline
\ \ \isacommand{fun}\isamarkupfalse%
\ uminus{\isacharunderscore}tag{\isacharunderscore}const\isanewline
\ \ \ \ \isakeyword{where}\ {\isacartoucheopen}uminus\ {\isacharparenleft}{\isasymtau}\isactrlsub c\isactrlsub s\isactrlsub t\ t{\isacharparenright}\ {\isacharequal}\ {\isasymtau}\isactrlsub c\isactrlsub s\isactrlsub t\ {\isacharparenleft}uminus\ t{\isacharparenright}{\isacartoucheclose}\isanewline
\isanewline
\isacommand{fun}\isamarkupfalse%
\ minus{\isacharunderscore}tag{\isacharunderscore}const\isanewline
\ \ \isakeyword{where}\ {\isacartoucheopen}minus\ {\isacharparenleft}{\isasymtau}\isactrlsub c\isactrlsub s\isactrlsub t\ t\isactrlsub {\isadigit{1}}{\isacharparenright}\ {\isacharparenleft}{\isasymtau}\isactrlsub c\isactrlsub s\isactrlsub t\ t\isactrlsub {\isadigit{2}}{\isacharparenright}\ {\isacharequal}\ {\isasymtau}\isactrlsub c\isactrlsub s\isactrlsub t\ {\isacharparenleft}minus\ t\isactrlsub {\isadigit{1}}\ t\isactrlsub {\isadigit{2}}{\isacharparenright}{\isacartoucheclose}\isanewline
\isanewline
\isacommand{definition}\isamarkupfalse%
\ {\isacartoucheopen}one{\isacharunderscore}tag{\isacharunderscore}const\ {\isasymequiv}\ {\isasymtau}\isactrlsub c\isactrlsub s\isactrlsub t\ {\isadigit{1}}{\isacartoucheclose}\isanewline
\isanewline
\isacommand{fun}\isamarkupfalse%
\ times{\isacharunderscore}tag{\isacharunderscore}const\isanewline
\ \ \isakeyword{where}\ {\isacartoucheopen}times\ {\isacharparenleft}{\isasymtau}\isactrlsub c\isactrlsub s\isactrlsub t\ t\isactrlsub {\isadigit{1}}{\isacharparenright}\ {\isacharparenleft}{\isasymtau}\isactrlsub c\isactrlsub s\isactrlsub t\ t\isactrlsub {\isadigit{2}}{\isacharparenright}\ {\isacharequal}\ {\isasymtau}\isactrlsub c\isactrlsub s\isactrlsub t\ {\isacharparenleft}times\ t\isactrlsub {\isadigit{1}}\ t\isactrlsub {\isadigit{2}}{\isacharparenright}{\isacartoucheclose}\isanewline
\isanewline
\isacommand{definition}\isamarkupfalse%
\ {\isacartoucheopen}zero{\isacharunderscore}tag{\isacharunderscore}const\ {\isasymequiv}\ {\isasymtau}\isactrlsub c\isactrlsub s\isactrlsub t\ {\isadigit{0}}{\isacartoucheclose}\isanewline
\isanewline
\isacommand{fun}\isamarkupfalse%
\ plus{\isacharunderscore}tag{\isacharunderscore}const\isanewline
\ \ \isakeyword{where}\ {\isacartoucheopen}plus\ {\isacharparenleft}{\isasymtau}\isactrlsub c\isactrlsub s\isactrlsub t\ t\isactrlsub {\isadigit{1}}{\isacharparenright}\ {\isacharparenleft}{\isasymtau}\isactrlsub c\isactrlsub s\isactrlsub t\ t\isactrlsub {\isadigit{2}}{\isacharparenright}\ {\isacharequal}\ {\isasymtau}\isactrlsub c\isactrlsub s\isactrlsub t\ {\isacharparenleft}plus\ t\isactrlsub {\isadigit{1}}\ t\isactrlsub {\isadigit{2}}{\isacharparenright}{\isacartoucheclose}\isanewline
\isanewline
\isacommand{instance}\isamarkupfalse%
%
\isadelimproof
\ %
\endisadelimproof
%
\isatagproof
\isacommand{proof}\isamarkupfalse%
%
\begin{isamarkuptext}%
Multiplication is associative.%
\end{isamarkuptext}\isamarkuptrue%
\ \ \isacommand{fix}\isamarkupfalse%
\ a{\isacharcolon}{\isacharcolon}{\isacartoucheopen}{\isacharprime}{\isasymtau}{\isacharcolon}{\isacharcolon}field\ tag{\isacharunderscore}const{\isacartoucheclose}\ \isakeyword{and}\ b{\isacharcolon}{\isacharcolon}{\isacartoucheopen}{\isacharprime}{\isasymtau}{\isacharcolon}{\isacharcolon}field\ tag{\isacharunderscore}const{\isacartoucheclose}\isanewline
\ \ \ \ \ \ \ \ \ \ \ \ \ \ \ \ \ \ \ \ \ \ \ \ \ \ \ \ \ \ \ \isakeyword{and}\ c{\isacharcolon}{\isacharcolon}{\isacartoucheopen}{\isacharprime}{\isasymtau}{\isacharcolon}{\isacharcolon}field\ tag{\isacharunderscore}const{\isacartoucheclose}\isanewline
\ \ \isacommand{obtain}\isamarkupfalse%
\ u\ \isakeyword{where}\ {\isacartoucheopen}a\ {\isacharequal}\ {\isasymtau}\isactrlsub c\isactrlsub s\isactrlsub t\ u{\isacartoucheclose}\ \isacommand{using}\isamarkupfalse%
\ tag{\isacharunderscore}const{\isachardot}exhaust\ \isacommand{by}\isamarkupfalse%
\ blast\isanewline
\ \ \isacommand{moreover}\isamarkupfalse%
\ \isacommand{obtain}\isamarkupfalse%
\ v\ \isakeyword{where}\ {\isacartoucheopen}b\ {\isacharequal}\ {\isasymtau}\isactrlsub c\isactrlsub s\isactrlsub t\ v{\isacartoucheclose}\ \isacommand{using}\isamarkupfalse%
\ tag{\isacharunderscore}const{\isachardot}exhaust\ \isacommand{by}\isamarkupfalse%
\ blast\isanewline
\ \ \isacommand{moreover}\isamarkupfalse%
\ \isacommand{obtain}\isamarkupfalse%
\ w\ \isakeyword{where}\ {\isacartoucheopen}c\ {\isacharequal}\ {\isasymtau}\isactrlsub c\isactrlsub s\isactrlsub t\ w{\isacartoucheclose}\ \isacommand{using}\isamarkupfalse%
\ tag{\isacharunderscore}const{\isachardot}exhaust\ \isacommand{by}\isamarkupfalse%
\ blast\isanewline
\ \ \isacommand{ultimately}\isamarkupfalse%
\ \isacommand{show}\isamarkupfalse%
\ {\isacartoucheopen}a\ {\isacharasterisk}\ b\ {\isacharasterisk}\ c\ {\isacharequal}\ a\ {\isacharasterisk}\ {\isacharparenleft}b\ {\isacharasterisk}\ c{\isacharparenright}{\isacartoucheclose}\isanewline
\ \ \ \ \isacommand{by}\isamarkupfalse%
\ {\isacharparenleft}simp\ add{\isacharcolon}\ TESL{\isachardot}times{\isacharunderscore}tag{\isacharunderscore}const{\isachardot}simps{\isacharparenright}\isanewline
\isacommand{next}\isamarkupfalse%
%
\begin{isamarkuptext}%
Multiplication is commutative.%
\end{isamarkuptext}\isamarkuptrue%
\ \ \isacommand{fix}\isamarkupfalse%
\ a{\isacharcolon}{\isacharcolon}{\isacartoucheopen}{\isacharprime}{\isasymtau}{\isacharcolon}{\isacharcolon}field\ tag{\isacharunderscore}const{\isacartoucheclose}\ \isakeyword{and}\ b{\isacharcolon}{\isacharcolon}{\isacartoucheopen}{\isacharprime}{\isasymtau}{\isacharcolon}{\isacharcolon}field\ tag{\isacharunderscore}const{\isacartoucheclose}\isanewline
\ \ \isacommand{obtain}\isamarkupfalse%
\ u\ \isakeyword{where}\ {\isacartoucheopen}a\ {\isacharequal}\ {\isasymtau}\isactrlsub c\isactrlsub s\isactrlsub t\ u{\isacartoucheclose}\ \isacommand{using}\isamarkupfalse%
\ tag{\isacharunderscore}const{\isachardot}exhaust\ \isacommand{by}\isamarkupfalse%
\ blast\isanewline
\ \ \isacommand{moreover}\isamarkupfalse%
\ \isacommand{obtain}\isamarkupfalse%
\ v\ \isakeyword{where}\ {\isacartoucheopen}b\ {\isacharequal}\ {\isasymtau}\isactrlsub c\isactrlsub s\isactrlsub t\ v{\isacartoucheclose}\ \isacommand{using}\isamarkupfalse%
\ tag{\isacharunderscore}const{\isachardot}exhaust\ \isacommand{by}\isamarkupfalse%
\ blast\isanewline
\ \ \isacommand{ultimately}\isamarkupfalse%
\ \isacommand{show}\isamarkupfalse%
\ {\isacartoucheopen}\ a\ {\isacharasterisk}\ b\ {\isacharequal}\ b\ {\isacharasterisk}\ a{\isacartoucheclose}\isanewline
\ \ \ \ \isacommand{by}\isamarkupfalse%
\ {\isacharparenleft}simp\ add{\isacharcolon}\ TESL{\isachardot}times{\isacharunderscore}tag{\isacharunderscore}const{\isachardot}simps{\isacharparenright}\isanewline
\isacommand{next}\isamarkupfalse%
%
\begin{isamarkuptext}%
One is neutral for multiplication.%
\end{isamarkuptext}\isamarkuptrue%
\ \ \isacommand{fix}\isamarkupfalse%
\ a{\isacharcolon}{\isacharcolon}{\isacartoucheopen}{\isacharprime}{\isasymtau}{\isacharcolon}{\isacharcolon}field\ tag{\isacharunderscore}const{\isacartoucheclose}\isanewline
\ \ \isacommand{obtain}\isamarkupfalse%
\ u\ \isakeyword{where}\ {\isacartoucheopen}a\ {\isacharequal}\ {\isasymtau}\isactrlsub c\isactrlsub s\isactrlsub t\ u{\isacartoucheclose}\ \isacommand{using}\isamarkupfalse%
\ tag{\isacharunderscore}const{\isachardot}exhaust\ \isacommand{by}\isamarkupfalse%
\ blast\isanewline
\ \ \isacommand{thus}\isamarkupfalse%
\ {\isacartoucheopen}{\isadigit{1}}\ {\isacharasterisk}\ a\ {\isacharequal}\ a{\isacartoucheclose}\isanewline
\ \ \ \ \isacommand{by}\isamarkupfalse%
\ {\isacharparenleft}simp\ add{\isacharcolon}\ TESL{\isachardot}times{\isacharunderscore}tag{\isacharunderscore}const{\isachardot}simps\ one{\isacharunderscore}tag{\isacharunderscore}const{\isacharunderscore}def{\isacharparenright}\isanewline
\isacommand{next}\isamarkupfalse%
%
\begin{isamarkuptext}%
Addition is associative.%
\end{isamarkuptext}\isamarkuptrue%
\ \ \isacommand{fix}\isamarkupfalse%
\ a{\isacharcolon}{\isacharcolon}{\isacartoucheopen}{\isacharprime}{\isasymtau}{\isacharcolon}{\isacharcolon}field\ tag{\isacharunderscore}const{\isacartoucheclose}\ \isakeyword{and}\ b{\isacharcolon}{\isacharcolon}{\isacartoucheopen}{\isacharprime}{\isasymtau}{\isacharcolon}{\isacharcolon}field\ tag{\isacharunderscore}const{\isacartoucheclose}\isanewline
\ \ \ \ \ \ \ \ \ \ \ \ \ \ \ \ \ \ \ \ \ \ \ \ \ \ \ \ \ \ \ \isakeyword{and}\ c{\isacharcolon}{\isacharcolon}{\isacartoucheopen}{\isacharprime}{\isasymtau}{\isacharcolon}{\isacharcolon}field\ tag{\isacharunderscore}const{\isacartoucheclose}\isanewline
\ \ \isacommand{obtain}\isamarkupfalse%
\ u\ \isakeyword{where}\ {\isacartoucheopen}a\ {\isacharequal}\ {\isasymtau}\isactrlsub c\isactrlsub s\isactrlsub t\ u{\isacartoucheclose}\ \isacommand{using}\isamarkupfalse%
\ tag{\isacharunderscore}const{\isachardot}exhaust\ \isacommand{by}\isamarkupfalse%
\ blast\isanewline
\ \ \isacommand{moreover}\isamarkupfalse%
\ \isacommand{obtain}\isamarkupfalse%
\ v\ \isakeyword{where}\ {\isacartoucheopen}b\ {\isacharequal}\ {\isasymtau}\isactrlsub c\isactrlsub s\isactrlsub t\ v{\isacartoucheclose}\ \isacommand{using}\isamarkupfalse%
\ tag{\isacharunderscore}const{\isachardot}exhaust\ \isacommand{by}\isamarkupfalse%
\ blast\isanewline
\ \ \isacommand{moreover}\isamarkupfalse%
\ \isacommand{obtain}\isamarkupfalse%
\ w\ \isakeyword{where}\ {\isacartoucheopen}c\ {\isacharequal}\ {\isasymtau}\isactrlsub c\isactrlsub s\isactrlsub t\ w{\isacartoucheclose}\ \isacommand{using}\isamarkupfalse%
\ tag{\isacharunderscore}const{\isachardot}exhaust\ \isacommand{by}\isamarkupfalse%
\ blast\isanewline
\ \ \isacommand{ultimately}\isamarkupfalse%
\ \isacommand{show}\isamarkupfalse%
\ {\isacartoucheopen}a\ {\isacharplus}\ b\ {\isacharplus}\ c\ {\isacharequal}\ a\ {\isacharplus}\ {\isacharparenleft}b\ {\isacharplus}\ c{\isacharparenright}{\isacartoucheclose}\isanewline
\ \ \ \ \isacommand{by}\isamarkupfalse%
\ {\isacharparenleft}simp\ add{\isacharcolon}\ TESL{\isachardot}plus{\isacharunderscore}tag{\isacharunderscore}const{\isachardot}simps{\isacharparenright}\isanewline
\isacommand{next}\isamarkupfalse%
%
\begin{isamarkuptext}%
Addition is commutative.%
\end{isamarkuptext}\isamarkuptrue%
\ \ \isacommand{fix}\isamarkupfalse%
\ a{\isacharcolon}{\isacharcolon}{\isacartoucheopen}{\isacharprime}{\isasymtau}{\isacharcolon}{\isacharcolon}field\ tag{\isacharunderscore}const{\isacartoucheclose}\ \isakeyword{and}\ b{\isacharcolon}{\isacharcolon}{\isacartoucheopen}{\isacharprime}{\isasymtau}{\isacharcolon}{\isacharcolon}field\ tag{\isacharunderscore}const{\isacartoucheclose}\isanewline
\ \ \isacommand{obtain}\isamarkupfalse%
\ u\ \isakeyword{where}\ {\isacartoucheopen}a\ {\isacharequal}\ {\isasymtau}\isactrlsub c\isactrlsub s\isactrlsub t\ u{\isacartoucheclose}\ \isacommand{using}\isamarkupfalse%
\ tag{\isacharunderscore}const{\isachardot}exhaust\ \isacommand{by}\isamarkupfalse%
\ blast\isanewline
\ \ \isacommand{moreover}\isamarkupfalse%
\ \isacommand{obtain}\isamarkupfalse%
\ v\ \isakeyword{where}\ {\isacartoucheopen}b\ {\isacharequal}\ {\isasymtau}\isactrlsub c\isactrlsub s\isactrlsub t\ v{\isacartoucheclose}\ \isacommand{using}\isamarkupfalse%
\ tag{\isacharunderscore}const{\isachardot}exhaust\ \isacommand{by}\isamarkupfalse%
\ blast\isanewline
\ \ \isacommand{ultimately}\isamarkupfalse%
\ \isacommand{show}\isamarkupfalse%
\ {\isacartoucheopen}a\ {\isacharplus}\ b\ {\isacharequal}\ b\ {\isacharplus}\ a{\isacartoucheclose}\isanewline
\ \ \ \ \isacommand{by}\isamarkupfalse%
\ {\isacharparenleft}simp\ add{\isacharcolon}\ TESL{\isachardot}plus{\isacharunderscore}tag{\isacharunderscore}const{\isachardot}simps{\isacharparenright}\isanewline
\isacommand{next}\isamarkupfalse%
%
\begin{isamarkuptext}%
Zero is neutral for addition.%
\end{isamarkuptext}\isamarkuptrue%
\ \ \isacommand{fix}\isamarkupfalse%
\ a{\isacharcolon}{\isacharcolon}{\isacartoucheopen}{\isacharprime}{\isasymtau}{\isacharcolon}{\isacharcolon}field\ tag{\isacharunderscore}const{\isacartoucheclose}\isanewline
\ \ \isacommand{obtain}\isamarkupfalse%
\ u\ \isakeyword{where}\ {\isacartoucheopen}a\ {\isacharequal}\ {\isasymtau}\isactrlsub c\isactrlsub s\isactrlsub t\ u{\isacartoucheclose}\ \isacommand{using}\isamarkupfalse%
\ tag{\isacharunderscore}const{\isachardot}exhaust\ \isacommand{by}\isamarkupfalse%
\ blast\isanewline
\ \ \isacommand{thus}\isamarkupfalse%
\ {\isacartoucheopen}{\isadigit{0}}\ {\isacharplus}\ a\ {\isacharequal}\ a{\isacartoucheclose}\isanewline
\ \ \ \ \isacommand{by}\isamarkupfalse%
\ {\isacharparenleft}simp\ add{\isacharcolon}\ TESL{\isachardot}plus{\isacharunderscore}tag{\isacharunderscore}const{\isachardot}simps\ zero{\isacharunderscore}tag{\isacharunderscore}const{\isacharunderscore}def{\isacharparenright}\isanewline
\isacommand{next}\isamarkupfalse%
%
\begin{isamarkuptext}%
The sum of an element and its opposite is zero.%
\end{isamarkuptext}\isamarkuptrue%
\ \ \isacommand{fix}\isamarkupfalse%
\ a{\isacharcolon}{\isacharcolon}{\isacartoucheopen}{\isacharprime}{\isasymtau}{\isacharcolon}{\isacharcolon}field\ tag{\isacharunderscore}const{\isacartoucheclose}\isanewline
\ \ \isacommand{obtain}\isamarkupfalse%
\ u\ \isakeyword{where}\ {\isacartoucheopen}a\ {\isacharequal}\ {\isasymtau}\isactrlsub c\isactrlsub s\isactrlsub t\ u{\isacartoucheclose}\ \isacommand{using}\isamarkupfalse%
\ tag{\isacharunderscore}const{\isachardot}exhaust\ \isacommand{by}\isamarkupfalse%
\ blast\isanewline
\ \ \isacommand{thus}\isamarkupfalse%
\ {\isacartoucheopen}{\isacharminus}a\ {\isacharplus}\ a\ {\isacharequal}\ {\isadigit{0}}{\isacartoucheclose}\isanewline
\ \ \ \ \isacommand{by}\isamarkupfalse%
\ {\isacharparenleft}simp\ add{\isacharcolon}\ TESL{\isachardot}plus{\isacharunderscore}tag{\isacharunderscore}const{\isachardot}simps\isanewline
\ \ \ \ \ \ \ \ \ \ \ \ \ \ \ \ \ \ TESL{\isachardot}uminus{\isacharunderscore}tag{\isacharunderscore}const{\isachardot}simps\isanewline
\ \ \ \ \ \ \ \ \ \ \ \ \ \ \ \ \ \ zero{\isacharunderscore}tag{\isacharunderscore}const{\isacharunderscore}def{\isacharparenright}\isanewline
\isacommand{next}\isamarkupfalse%
%
\begin{isamarkuptext}%
Subtraction is adding the opposite.%
\end{isamarkuptext}\isamarkuptrue%
\ \ \isacommand{fix}\isamarkupfalse%
\ a{\isacharcolon}{\isacharcolon}{\isacartoucheopen}{\isacharprime}{\isasymtau}{\isacharcolon}{\isacharcolon}field\ tag{\isacharunderscore}const{\isacartoucheclose}\ \isakeyword{and}\ b{\isacharcolon}{\isacharcolon}{\isacartoucheopen}{\isacharprime}{\isasymtau}{\isacharcolon}{\isacharcolon}field\ tag{\isacharunderscore}const{\isacartoucheclose}\isanewline
\ \ \isacommand{obtain}\isamarkupfalse%
\ u\ \isakeyword{where}\ {\isacartoucheopen}a\ {\isacharequal}\ {\isasymtau}\isactrlsub c\isactrlsub s\isactrlsub t\ u{\isacartoucheclose}\ \isacommand{using}\isamarkupfalse%
\ tag{\isacharunderscore}const{\isachardot}exhaust\ \isacommand{by}\isamarkupfalse%
\ blast\isanewline
\ \ \isacommand{moreover}\isamarkupfalse%
\ \isacommand{obtain}\isamarkupfalse%
\ v\ \isakeyword{where}\ {\isacartoucheopen}b\ {\isacharequal}\ {\isasymtau}\isactrlsub c\isactrlsub s\isactrlsub t\ v{\isacartoucheclose}\ \isacommand{using}\isamarkupfalse%
\ tag{\isacharunderscore}const{\isachardot}exhaust\ \isacommand{by}\isamarkupfalse%
\ blast\isanewline
\ \ \isacommand{ultimately}\isamarkupfalse%
\ \isacommand{show}\isamarkupfalse%
\ {\isacartoucheopen}a\ {\isacharminus}\ b\ {\isacharequal}\ a\ {\isacharplus}\ {\isacharminus}b{\isacartoucheclose}\isanewline
\ \ \ \ \isacommand{by}\isamarkupfalse%
\ {\isacharparenleft}simp\ add{\isacharcolon}\ TESL{\isachardot}minus{\isacharunderscore}tag{\isacharunderscore}const{\isachardot}simps\isanewline
\ \ \ \ \ \ \ \ \ \ \ \ \ \ \ \ \ \ TESL{\isachardot}plus{\isacharunderscore}tag{\isacharunderscore}const{\isachardot}simps\isanewline
\ \ \ \ \ \ \ \ \ \ \ \ \ \ \ \ \ \ TESL{\isachardot}uminus{\isacharunderscore}tag{\isacharunderscore}const{\isachardot}simps{\isacharparenright}\isanewline
\isacommand{next}\isamarkupfalse%
%
\begin{isamarkuptext}%
Distributive property of multiplication over addition.%
\end{isamarkuptext}\isamarkuptrue%
\ \ \isacommand{fix}\isamarkupfalse%
\ a{\isacharcolon}{\isacharcolon}{\isacartoucheopen}{\isacharprime}{\isasymtau}{\isacharcolon}{\isacharcolon}field\ tag{\isacharunderscore}const{\isacartoucheclose}\ \isakeyword{and}\ b{\isacharcolon}{\isacharcolon}{\isacartoucheopen}{\isacharprime}{\isasymtau}{\isacharcolon}{\isacharcolon}field\ tag{\isacharunderscore}const{\isacartoucheclose}\isanewline
\ \ \ \ \ \ \ \ \ \ \ \ \ \ \ \ \ \ \ \ \ \ \ \ \ \ \ \ \ \ \ \isakeyword{and}\ c{\isacharcolon}{\isacharcolon}{\isacartoucheopen}{\isacharprime}{\isasymtau}{\isacharcolon}{\isacharcolon}field\ tag{\isacharunderscore}const{\isacartoucheclose}\isanewline
\ \ \isacommand{obtain}\isamarkupfalse%
\ u\ \isakeyword{where}\ {\isacartoucheopen}a\ {\isacharequal}\ {\isasymtau}\isactrlsub c\isactrlsub s\isactrlsub t\ u{\isacartoucheclose}\ \isacommand{using}\isamarkupfalse%
\ tag{\isacharunderscore}const{\isachardot}exhaust\ \isacommand{by}\isamarkupfalse%
\ blast\isanewline
\ \ \isacommand{moreover}\isamarkupfalse%
\ \isacommand{obtain}\isamarkupfalse%
\ v\ \isakeyword{where}\ {\isacartoucheopen}b\ {\isacharequal}\ {\isasymtau}\isactrlsub c\isactrlsub s\isactrlsub t\ v{\isacartoucheclose}\ \isacommand{using}\isamarkupfalse%
\ tag{\isacharunderscore}const{\isachardot}exhaust\ \isacommand{by}\isamarkupfalse%
\ blast\isanewline
\ \ \isacommand{moreover}\isamarkupfalse%
\ \isacommand{obtain}\isamarkupfalse%
\ w\ \isakeyword{where}\ {\isacartoucheopen}c\ {\isacharequal}\ {\isasymtau}\isactrlsub c\isactrlsub s\isactrlsub t\ w{\isacartoucheclose}\ \isacommand{using}\isamarkupfalse%
\ tag{\isacharunderscore}const{\isachardot}exhaust\ \isacommand{by}\isamarkupfalse%
\ blast\isanewline
\ \ \isacommand{ultimately}\isamarkupfalse%
\ \isacommand{show}\isamarkupfalse%
\ {\isacartoucheopen}{\isacharparenleft}a\ {\isacharplus}\ b{\isacharparenright}\ {\isacharasterisk}\ c\ {\isacharequal}\ a\ {\isacharasterisk}\ c\ {\isacharplus}\ b\ {\isacharasterisk}\ c{\isacartoucheclose}\isanewline
\ \ \ \ \isacommand{by}\isamarkupfalse%
\ {\isacharparenleft}simp\ add{\isacharcolon}\ TESL{\isachardot}plus{\isacharunderscore}tag{\isacharunderscore}const{\isachardot}simps\isanewline
\ \ \ \ \ \ \ \ \ \ \ \ \ \ \ \ \ \ TESL{\isachardot}times{\isacharunderscore}tag{\isacharunderscore}const{\isachardot}simps\isanewline
\ \ \ \ \ \ \ \ \ \ \ \ \ \ \ \ \ \ ring{\isacharunderscore}class{\isachardot}ring{\isacharunderscore}distribs{\isacharparenleft}{\isadigit{2}}{\isacharparenright}{\isacharparenright}\isanewline
\isacommand{next}\isamarkupfalse%
%
\begin{isamarkuptext}%
The neutral elements are distinct.%
\end{isamarkuptext}\isamarkuptrue%
\ \ \isacommand{show}\isamarkupfalse%
\ {\isacartoucheopen}{\isacharparenleft}{\isadigit{0}}{\isacharcolon}{\isacharcolon}{\isacharparenleft}{\isacharprime}{\isasymtau}{\isacharcolon}{\isacharcolon}field\ tag{\isacharunderscore}const{\isacharparenright}{\isacharparenright}\ {\isasymnoteq}\ {\isadigit{1}}{\isacartoucheclose}\isanewline
\ \ \ \ \isacommand{by}\isamarkupfalse%
\ {\isacharparenleft}simp\ add{\isacharcolon}\ one{\isacharunderscore}tag{\isacharunderscore}const{\isacharunderscore}def\ zero{\isacharunderscore}tag{\isacharunderscore}const{\isacharunderscore}def{\isacharparenright}\isanewline
\isacommand{next}\isamarkupfalse%
%
\begin{isamarkuptext}%
The product of an element and its inverse is 1.%
\end{isamarkuptext}\isamarkuptrue%
\ \ \isacommand{fix}\isamarkupfalse%
\ a{\isacharcolon}{\isacharcolon}{\isacartoucheopen}{\isacharprime}{\isasymtau}{\isacharcolon}{\isacharcolon}field\ tag{\isacharunderscore}const{\isacartoucheclose}\ \isacommand{assume}\isamarkupfalse%
\ h{\isacharcolon}{\isacartoucheopen}a\ {\isasymnoteq}\ {\isadigit{0}}{\isacartoucheclose}\isanewline
\ \ \isacommand{obtain}\isamarkupfalse%
\ u\ \isakeyword{where}\ {\isacartoucheopen}a\ {\isacharequal}\ {\isasymtau}\isactrlsub c\isactrlsub s\isactrlsub t\ u{\isacartoucheclose}\ \isacommand{using}\isamarkupfalse%
\ tag{\isacharunderscore}const{\isachardot}exhaust\ \isacommand{by}\isamarkupfalse%
\ blast\isanewline
\ \ \isacommand{moreover}\isamarkupfalse%
\ \isacommand{with}\isamarkupfalse%
\ h\ \isacommand{have}\isamarkupfalse%
\ {\isacartoucheopen}u\ {\isasymnoteq}\ {\isadigit{0}}{\isacartoucheclose}\ \isacommand{by}\isamarkupfalse%
\ {\isacharparenleft}simp\ add{\isacharcolon}\ zero{\isacharunderscore}tag{\isacharunderscore}const{\isacharunderscore}def{\isacharparenright}\isanewline
\ \ \isacommand{ultimately}\isamarkupfalse%
\ \isacommand{show}\isamarkupfalse%
\ {\isacartoucheopen}inverse\ a\ {\isacharasterisk}\ a\ {\isacharequal}\ {\isadigit{1}}{\isacartoucheclose}\isanewline
\ \ \ \ \isacommand{by}\isamarkupfalse%
\ {\isacharparenleft}simp\ add{\isacharcolon}\ TESL{\isachardot}inverse{\isacharunderscore}tag{\isacharunderscore}const{\isachardot}simps\isanewline
\ \ \ \ \ \ \ \ \ \ \ \ \ \ \ \ \ \ TESL{\isachardot}times{\isacharunderscore}tag{\isacharunderscore}const{\isachardot}simps\isanewline
\ \ \ \ \ \ \ \ \ \ \ \ \ \ \ \ \ \ one{\isacharunderscore}tag{\isacharunderscore}const{\isacharunderscore}def{\isacharparenright}\isanewline
\isacommand{next}\isamarkupfalse%
%
\begin{isamarkuptext}%
Dividing is multiplying by the inverse.%
\end{isamarkuptext}\isamarkuptrue%
\ \ \isacommand{fix}\isamarkupfalse%
\ a{\isacharcolon}{\isacharcolon}{\isacartoucheopen}{\isacharprime}{\isasymtau}{\isacharcolon}{\isacharcolon}field\ tag{\isacharunderscore}const{\isacartoucheclose}\ \isakeyword{and}\ b{\isacharcolon}{\isacharcolon}{\isacartoucheopen}{\isacharprime}{\isasymtau}{\isacharcolon}{\isacharcolon}field\ tag{\isacharunderscore}const{\isacartoucheclose}\isanewline
\ \ \isacommand{obtain}\isamarkupfalse%
\ u\ \isakeyword{where}\ {\isacartoucheopen}a\ {\isacharequal}\ {\isasymtau}\isactrlsub c\isactrlsub s\isactrlsub t\ u{\isacartoucheclose}\ \isacommand{using}\isamarkupfalse%
\ tag{\isacharunderscore}const{\isachardot}exhaust\ \isacommand{by}\isamarkupfalse%
\ blast\isanewline
\ \ \isacommand{moreover}\isamarkupfalse%
\ \isacommand{obtain}\isamarkupfalse%
\ v\ \isakeyword{where}\ {\isacartoucheopen}b\ {\isacharequal}\ {\isasymtau}\isactrlsub c\isactrlsub s\isactrlsub t\ v{\isacartoucheclose}\ \isacommand{using}\isamarkupfalse%
\ tag{\isacharunderscore}const{\isachardot}exhaust\ \isacommand{by}\isamarkupfalse%
\ blast\isanewline
\ \ \isacommand{ultimately}\isamarkupfalse%
\ \isacommand{show}\isamarkupfalse%
\ {\isacartoucheopen}a\ div\ b\ {\isacharequal}\ a\ {\isacharasterisk}\ inverse\ b{\isacartoucheclose}\isanewline
\ \ \ \ \isacommand{by}\isamarkupfalse%
\ {\isacharparenleft}simp\ add{\isacharcolon}\ TESL{\isachardot}divide{\isacharunderscore}tag{\isacharunderscore}const{\isachardot}simps\isanewline
\ \ \ \ \ \ \ \ \ \ \ \ \ \ \ \ \ \ TESL{\isachardot}inverse{\isacharunderscore}tag{\isacharunderscore}const{\isachardot}simps\isanewline
\ \ \ \ \ \ \ \ \ \ \ \ \ \ \ \ \ \ TESL{\isachardot}times{\isacharunderscore}tag{\isacharunderscore}const{\isachardot}simps\isanewline
\ \ \ \ \ \ \ \ \ \ \ \ \ \ \ \ \ \ divide{\isacharunderscore}inverse{\isacharparenright}\isanewline
\isacommand{next}\isamarkupfalse%
%
\begin{isamarkuptext}%
Zero is its own inverse.%
\end{isamarkuptext}\isamarkuptrue%
\ \ \isacommand{show}\isamarkupfalse%
\ {\isacartoucheopen}inverse\ {\isacharparenleft}{\isadigit{0}}{\isacharcolon}{\isacharcolon}{\isacharparenleft}{\isacharprime}{\isasymtau}{\isacharcolon}{\isacharcolon}field\ tag{\isacharunderscore}const{\isacharparenright}{\isacharparenright}\ {\isacharequal}\ {\isadigit{0}}{\isacartoucheclose}\isanewline
\ \ \ \ \isacommand{by}\isamarkupfalse%
\ {\isacharparenleft}simp\ add{\isacharcolon}\ TESL{\isachardot}inverse{\isacharunderscore}tag{\isacharunderscore}const{\isachardot}simps\ zero{\isacharunderscore}tag{\isacharunderscore}const{\isacharunderscore}def{\isacharparenright}\isanewline
\isacommand{qed}\isamarkupfalse%
%
\endisatagproof
{\isafoldproof}%
%
\isadelimproof
%
\endisadelimproof
\isanewline
\isanewline
\isacommand{end}\isamarkupfalse%
%
\begin{isamarkuptext}%
For comparing dates (which are represented by tags) on clocks, we need an order on tags.%
\end{isamarkuptext}\isamarkuptrue%
\isacommand{instantiation}\isamarkupfalse%
\ tag{\isacharunderscore}const\ {\isacharcolon}{\isacharcolon}\ {\isacharparenleft}order{\isacharparenright}order\isanewline
\isakeyword{begin}\isanewline
\ \ \isacommand{inductive}\isamarkupfalse%
\ less{\isacharunderscore}eq{\isacharunderscore}tag{\isacharunderscore}const\ {\isacharcolon}{\isacharcolon}\ {\isacartoucheopen}{\isacharprime}a\ tag{\isacharunderscore}const\ {\isasymRightarrow}\ {\isacharprime}a\ tag{\isacharunderscore}const\ {\isasymRightarrow}\ bool{\isacartoucheclose}\isanewline
\ \ \isakeyword{where}\isanewline
\ \ \ \ Int{\isacharunderscore}less{\isacharunderscore}eq{\isacharbrackleft}simp{\isacharbrackright}{\isacharcolon}\ \ \ \ \ \ {\isacartoucheopen}n\ {\isasymle}\ m\ {\isasymLongrightarrow}\ {\isacharparenleft}TConst\ n{\isacharparenright}\ {\isasymle}\ {\isacharparenleft}TConst\ m{\isacharparenright}{\isacartoucheclose}\isanewline
\isanewline
\ \ \isacommand{definition}\isamarkupfalse%
\ less{\isacharunderscore}tag{\isacharcolon}\ {\isacartoucheopen}{\isacharparenleft}x{\isacharcolon}{\isacharcolon}{\isacharprime}a\ tag{\isacharunderscore}const{\isacharparenright}\ {\isacharless}\ y\ {\isasymlongleftrightarrow}\ {\isacharparenleft}x\ {\isasymle}\ y{\isacharparenright}\ {\isasymand}\ {\isacharparenleft}x\ {\isasymnoteq}\ y{\isacharparenright}{\isacartoucheclose}\isanewline
\isanewline
\ \ \isacommand{instance}\isamarkupfalse%
%
\isadelimproof
\ %
\endisadelimproof
%
\isatagproof
\isacommand{proof}\isamarkupfalse%
\isanewline
\ \ \ \ \isacommand{show}\isamarkupfalse%
\ {\isacartoucheopen}{\isasymAnd}x\ y\ {\isacharcolon}{\isacharcolon}\ {\isacharprime}a\ tag{\isacharunderscore}const{\isachardot}\ {\isacharparenleft}x\ {\isacharless}\ y{\isacharparenright}\ {\isacharequal}\ {\isacharparenleft}x\ {\isasymle}\ y\ {\isasymand}\ {\isasymnot}\ y\ {\isasymle}\ x{\isacharparenright}{\isacartoucheclose}\isanewline
\ \ \ \ \ \ \isacommand{using}\isamarkupfalse%
\ less{\isacharunderscore}eq{\isacharunderscore}tag{\isacharunderscore}const{\isachardot}simps\ less{\isacharunderscore}tag\ \isacommand{by}\isamarkupfalse%
\ auto\isanewline
\ \ \isacommand{next}\isamarkupfalse%
\isanewline
\ \ \ \ \isacommand{fix}\isamarkupfalse%
\ x{\isacharcolon}{\isacharcolon}{\isacartoucheopen}{\isacharprime}a\ tag{\isacharunderscore}const{\isacartoucheclose}\isanewline
\ \ \ \ \isacommand{from}\isamarkupfalse%
\ tag{\isacharunderscore}const{\isachardot}exhaust\ \isacommand{obtain}\isamarkupfalse%
\ x\isactrlsub {\isadigit{0}}{\isacharcolon}{\isacharcolon}{\isacharprime}a\ \isakeyword{where}\ {\isacartoucheopen}x\ {\isacharequal}\ TConst\ x\isactrlsub {\isadigit{0}}{\isacartoucheclose}\ \isacommand{by}\isamarkupfalse%
\ blast\isanewline
\ \ \ \ \isacommand{with}\isamarkupfalse%
\ Int{\isacharunderscore}less{\isacharunderscore}eq\ \isacommand{show}\isamarkupfalse%
\ {\isacartoucheopen}x\ {\isasymle}\ x{\isacartoucheclose}\ \isacommand{by}\isamarkupfalse%
\ simp\isanewline
\ \ \isacommand{next}\isamarkupfalse%
\isanewline
\ \ \ \ \isacommand{show}\isamarkupfalse%
\ {\isacartoucheopen}{\isasymAnd}x\ y\ z\ \ {\isacharcolon}{\isacharcolon}\ {\isacharprime}a\ tag{\isacharunderscore}const{\isachardot}\ x\ {\isasymle}\ y\ {\isasymLongrightarrow}\ y\ {\isasymle}\ z\ {\isasymLongrightarrow}\ x\ {\isasymle}\ z{\isacartoucheclose}\isanewline
\ \ \ \ \ \ \isacommand{using}\isamarkupfalse%
\ less{\isacharunderscore}eq{\isacharunderscore}tag{\isacharunderscore}const{\isachardot}simps\ \isacommand{by}\isamarkupfalse%
\ auto\isanewline
\ \ \isacommand{next}\isamarkupfalse%
\isanewline
\ \ \ \ \isacommand{show}\isamarkupfalse%
\ {\isacartoucheopen}{\isasymAnd}x\ y\ \ {\isacharcolon}{\isacharcolon}\ {\isacharprime}a\ tag{\isacharunderscore}const{\isachardot}\ x\ {\isasymle}\ y\ {\isasymLongrightarrow}\ y\ {\isasymle}\ x\ {\isasymLongrightarrow}\ x\ {\isacharequal}\ y{\isacartoucheclose}\isanewline
\ \ \ \ \ \ \isacommand{using}\isamarkupfalse%
\ less{\isacharunderscore}eq{\isacharunderscore}tag{\isacharunderscore}const{\isachardot}simps\ \isacommand{by}\isamarkupfalse%
\ auto\isanewline
\ \ \isacommand{qed}\isamarkupfalse%
%
\endisatagproof
{\isafoldproof}%
%
\isadelimproof
%
\endisadelimproof
\isanewline
\isanewline
\isacommand{end}\isamarkupfalse%
%
\begin{isamarkuptext}%
For ensuring that time does never flow backwards, we need a total order on tags.%
\end{isamarkuptext}\isamarkuptrue%
\isacommand{instantiation}\isamarkupfalse%
\ tag{\isacharunderscore}const\ {\isacharcolon}{\isacharcolon}\ {\isacharparenleft}linorder{\isacharparenright}linorder\isanewline
\isakeyword{begin}\isanewline
\ \ \isacommand{instance}\isamarkupfalse%
%
\isadelimproof
\ %
\endisadelimproof
%
\isatagproof
\isacommand{proof}\isamarkupfalse%
\isanewline
\ \ \ \ \isacommand{fix}\isamarkupfalse%
\ x{\isacharcolon}{\isacharcolon}{\isacartoucheopen}{\isacharprime}a\ tag{\isacharunderscore}const{\isacartoucheclose}\ \isakeyword{and}\ y{\isacharcolon}{\isacharcolon}{\isacartoucheopen}{\isacharprime}a\ tag{\isacharunderscore}const{\isacartoucheclose}\isanewline
\ \ \ \ \isacommand{from}\isamarkupfalse%
\ tag{\isacharunderscore}const{\isachardot}exhaust\ \isacommand{obtain}\isamarkupfalse%
\ x\isactrlsub {\isadigit{0}}{\isacharcolon}{\isacharcolon}{\isacharprime}a\ \isakeyword{where}\ {\isacartoucheopen}x\ {\isacharequal}\ TConst\ x\isactrlsub {\isadigit{0}}{\isacartoucheclose}\ \isacommand{by}\isamarkupfalse%
\ blast\isanewline
\ \ \ \ \isacommand{moreover}\isamarkupfalse%
\ \isacommand{from}\isamarkupfalse%
\ tag{\isacharunderscore}const{\isachardot}exhaust\ \isacommand{obtain}\isamarkupfalse%
\ y\isactrlsub {\isadigit{0}}{\isacharcolon}{\isacharcolon}{\isacharprime}a\ \isakeyword{where}\ {\isacartoucheopen}y\ {\isacharequal}\ TConst\ y\isactrlsub {\isadigit{0}}{\isacartoucheclose}\ \isacommand{by}\isamarkupfalse%
\ blast\isanewline
\ \ \ \ \isacommand{ultimately}\isamarkupfalse%
\ \isacommand{show}\isamarkupfalse%
\ {\isacartoucheopen}x\ {\isasymle}\ y\ {\isasymor}\ y\ {\isasymle}\ x{\isacartoucheclose}\ \isacommand{using}\isamarkupfalse%
\ less{\isacharunderscore}eq{\isacharunderscore}tag{\isacharunderscore}const{\isachardot}simps\ \isacommand{by}\isamarkupfalse%
\ fastforce\isanewline
\ \ \isacommand{qed}\isamarkupfalse%
%
\endisatagproof
{\isafoldproof}%
%
\isadelimproof
%
\endisadelimproof
\isanewline
\isanewline
\isacommand{end}\isamarkupfalse%
\isanewline
%
\isadelimtheory
\isanewline
%
\endisadelimtheory
%
\isatagtheory
\isacommand{end}\isamarkupfalse%
%
\endisatagtheory
{\isafoldtheory}%
%
\isadelimtheory
%
\endisadelimtheory
%
\end{isabellebody}%
\endinput
%:%file=~/MEGAsync/code/these/hygge/src/TESL.thy%:%
%:%6=2%:%
%:%14=4%:%
%:%15=4%:%
%:%16=5%:%
%:%17=6%:%
%:%18=7%:%
%:%32=9%:%
%:%44=11%:%
%:%53=14%:%
%:%65=16%:%
%:%69=17%:%
%:%71=18%:%
%:%73=19%:%
%:%76=22%:%
%:%77=22%:%
%:%78=23%:%
%:%79=23%:%
%:%80=24%:%
%:%81=25%:%
%:%82=25%:%
%:%84=28%:%
%:%85=29%:%
%:%87=31%:%
%:%88=31%:%
%:%89=32%:%
%:%90=33%:%
%:%91=33%:%
%:%92=34%:%
%:%99=36%:%
%:%111=38%:%
%:%113=40%:%
%:%114=40%:%
%:%115=41%:%
%:%116=42%:%
%:%117=43%:%
%:%118=44%:%
%:%119=45%:%
%:%120=46%:%
%:%121=47%:%
%:%122=48%:%
%:%123=49%:%
%:%124=50%:%
%:%125=51%:%
%:%126=52%:%
%:%128=53%:%
%:%129=54%:%
%:%130=54%:%
%:%131=55%:%
%:%133=58%:%
%:%134=59%:%
%:%136=61%:%
%:%137=61%:%
%:%138=62%:%
%:%139=63%:%
%:%141=66%:%
%:%142=67%:%
%:%144=69%:%
%:%145=69%:%
%:%146=70%:%
%:%147=71%:%
%:%148=71%:%
%:%149=72%:%
%:%150=73%:%
%:%152=76%:%
%:%153=77%:%
%:%155=79%:%
%:%156=79%:%
%:%157=80%:%
%:%158=81%:%
%:%159=82%:%
%:%161=85%:%
%:%163=87%:%
%:%164=87%:%
%:%165=88%:%
%:%166=89%:%
%:%175=93%:%
%:%187=95%:%
%:%188=96%:%
%:%189=97%:%
%:%191=99%:%
%:%192=99%:%
%:%193=100%:%
%:%194=101%:%
%:%195=101%:%
%:%196=102%:%
%:%197=103%:%
%:%198=104%:%
%:%199=104%:%
%:%200=105%:%
%:%201=106%:%
%:%202=107%:%
%:%203=107%:%
%:%204=108%:%
%:%205=109%:%
%:%206=110%:%
%:%207=110%:%
%:%208=111%:%
%:%209=112%:%
%:%210=113%:%
%:%211=113%:%
%:%212=114%:%
%:%213=115%:%
%:%214=115%:%
%:%215=116%:%
%:%216=117%:%
%:%217=118%:%
%:%218=118%:%
%:%219=119%:%
%:%220=120%:%
%:%221=120%:%
%:%222=121%:%
%:%223=122%:%
%:%224=123%:%
%:%227=123%:%
%:%231=123%:%
%:%234=124%:%
%:%236=125%:%
%:%237=125%:%
%:%238=126%:%
%:%239=127%:%
%:%240=127%:%
%:%241=127%:%
%:%242=127%:%
%:%243=128%:%
%:%244=128%:%
%:%245=128%:%
%:%246=128%:%
%:%247=128%:%
%:%248=129%:%
%:%249=129%:%
%:%250=129%:%
%:%251=129%:%
%:%252=129%:%
%:%253=130%:%
%:%254=130%:%
%:%255=130%:%
%:%256=131%:%
%:%257=131%:%
%:%258=132%:%
%:%261=133%:%
%:%263=134%:%
%:%264=134%:%
%:%265=135%:%
%:%266=135%:%
%:%267=135%:%
%:%268=135%:%
%:%269=136%:%
%:%270=136%:%
%:%271=136%:%
%:%272=136%:%
%:%273=136%:%
%:%274=137%:%
%:%275=137%:%
%:%276=137%:%
%:%277=138%:%
%:%278=138%:%
%:%279=139%:%
%:%282=140%:%
%:%284=141%:%
%:%285=141%:%
%:%286=142%:%
%:%287=142%:%
%:%288=142%:%
%:%289=142%:%
%:%290=143%:%
%:%291=143%:%
%:%292=144%:%
%:%293=144%:%
%:%294=145%:%
%:%297=146%:%
%:%299=147%:%
%:%300=147%:%
%:%301=148%:%
%:%302=149%:%
%:%303=149%:%
%:%304=149%:%
%:%305=149%:%
%:%306=150%:%
%:%307=150%:%
%:%308=150%:%
%:%309=150%:%
%:%310=150%:%
%:%311=151%:%
%:%312=151%:%
%:%313=151%:%
%:%314=151%:%
%:%315=151%:%
%:%316=152%:%
%:%317=152%:%
%:%318=152%:%
%:%319=153%:%
%:%320=153%:%
%:%321=154%:%
%:%324=155%:%
%:%326=156%:%
%:%327=156%:%
%:%328=157%:%
%:%329=157%:%
%:%330=157%:%
%:%331=157%:%
%:%332=158%:%
%:%333=158%:%
%:%334=158%:%
%:%335=158%:%
%:%336=158%:%
%:%337=159%:%
%:%338=159%:%
%:%339=159%:%
%:%340=160%:%
%:%341=160%:%
%:%342=161%:%
%:%345=162%:%
%:%347=163%:%
%:%348=163%:%
%:%349=164%:%
%:%350=164%:%
%:%351=164%:%
%:%352=164%:%
%:%353=165%:%
%:%354=165%:%
%:%355=166%:%
%:%356=166%:%
%:%357=167%:%
%:%360=168%:%
%:%362=169%:%
%:%363=169%:%
%:%364=170%:%
%:%365=170%:%
%:%366=170%:%
%:%367=170%:%
%:%368=171%:%
%:%369=171%:%
%:%370=172%:%
%:%371=172%:%
%:%372=173%:%
%:%373=174%:%
%:%374=175%:%
%:%377=176%:%
%:%379=177%:%
%:%380=177%:%
%:%381=178%:%
%:%382=178%:%
%:%383=178%:%
%:%384=178%:%
%:%385=179%:%
%:%386=179%:%
%:%387=179%:%
%:%388=179%:%
%:%389=179%:%
%:%390=180%:%
%:%391=180%:%
%:%392=180%:%
%:%393=181%:%
%:%394=181%:%
%:%395=182%:%
%:%396=183%:%
%:%397=184%:%
%:%400=185%:%
%:%402=186%:%
%:%403=186%:%
%:%404=187%:%
%:%405=188%:%
%:%406=188%:%
%:%407=188%:%
%:%408=188%:%
%:%409=189%:%
%:%410=189%:%
%:%411=189%:%
%:%412=189%:%
%:%413=189%:%
%:%414=190%:%
%:%415=190%:%
%:%416=190%:%
%:%417=190%:%
%:%418=190%:%
%:%419=191%:%
%:%420=191%:%
%:%421=191%:%
%:%422=192%:%
%:%423=192%:%
%:%424=193%:%
%:%425=194%:%
%:%426=195%:%
%:%429=196%:%
%:%431=197%:%
%:%432=197%:%
%:%433=198%:%
%:%434=198%:%
%:%435=199%:%
%:%438=200%:%
%:%440=201%:%
%:%441=201%:%
%:%442=201%:%
%:%443=202%:%
%:%444=202%:%
%:%445=202%:%
%:%446=202%:%
%:%447=203%:%
%:%448=203%:%
%:%449=203%:%
%:%450=203%:%
%:%451=203%:%
%:%452=204%:%
%:%453=204%:%
%:%454=204%:%
%:%455=205%:%
%:%456=205%:%
%:%457=206%:%
%:%458=207%:%
%:%459=208%:%
%:%462=209%:%
%:%464=210%:%
%:%465=210%:%
%:%466=211%:%
%:%467=211%:%
%:%468=211%:%
%:%469=211%:%
%:%470=212%:%
%:%471=212%:%
%:%472=212%:%
%:%473=212%:%
%:%474=212%:%
%:%475=213%:%
%:%476=213%:%
%:%477=213%:%
%:%478=214%:%
%:%479=214%:%
%:%480=215%:%
%:%481=216%:%
%:%482=217%:%
%:%483=218%:%
%:%486=219%:%
%:%488=220%:%
%:%489=220%:%
%:%490=221%:%
%:%491=221%:%
%:%492=222%:%
%:%500=222%:%
%:%501=223%:%
%:%502=224%:%
%:%505=227%:%
%:%507=230%:%
%:%508=230%:%
%:%509=231%:%
%:%510=232%:%
%:%511=232%:%
%:%512=233%:%
%:%513=234%:%
%:%514=235%:%
%:%515=236%:%
%:%516=236%:%
%:%517=237%:%
%:%518=238%:%
%:%521=238%:%
%:%525=238%:%
%:%526=238%:%
%:%527=239%:%
%:%528=239%:%
%:%529=240%:%
%:%530=240%:%
%:%531=240%:%
%:%532=241%:%
%:%533=241%:%
%:%534=242%:%
%:%535=242%:%
%:%536=243%:%
%:%537=243%:%
%:%538=243%:%
%:%539=243%:%
%:%540=244%:%
%:%541=244%:%
%:%542=244%:%
%:%543=244%:%
%:%544=245%:%
%:%545=245%:%
%:%546=246%:%
%:%547=246%:%
%:%548=247%:%
%:%549=247%:%
%:%550=247%:%
%:%551=248%:%
%:%552=248%:%
%:%553=249%:%
%:%554=249%:%
%:%555=250%:%
%:%556=250%:%
%:%557=250%:%
%:%558=251%:%
%:%566=251%:%
%:%567=252%:%
%:%568=253%:%
%:%571=256%:%
%:%573=258%:%
%:%574=258%:%
%:%575=259%:%
%:%576=260%:%
%:%579=260%:%
%:%583=260%:%
%:%584=260%:%
%:%585=261%:%
%:%586=261%:%
%:%587=262%:%
%:%588=262%:%
%:%589=262%:%
%:%590=262%:%
%:%591=263%:%
%:%592=263%:%
%:%593=263%:%
%:%594=263%:%
%:%595=263%:%
%:%596=264%:%
%:%597=264%:%
%:%598=264%:%
%:%599=264%:%
%:%600=264%:%
%:%601=265%:%
%:%609=265%:%
%:%610=266%:%
%:%611=267%:%
%:%612=267%:%
%:%615=268%:%
%:%620=269%:%